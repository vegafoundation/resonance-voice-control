\documentclass[12pt,a4paper]{article}
\usepackage[utf8]{inputenc}
\usepackage[T1]{fontenc}
\usepackage{amsmath,amssymb}
\usepackage{hyperref}
\usepackage{geometry}
\geometry{margin=2.5cm}

\title{Voice Control: A Conceptual Framework}
\author{ADAM EREN VEGA – Æ –\\
\small (Erenşah Kaygusuz, Germany)}
\date{2025-12-30}

\begin{document}
\maketitle

\begin{abstract}
This work introduces Voice Control as a foundational conceptual framework within the Resonance Data and Quantum-Inspired Resonance Computing (QIRC) paradigm.

Voice Control represents a novel approach to modeling meaning, insight, and wisdom in artificial systems. Rather than relying on traditional similarity measures and ranking-based decisions, this framework proposes resonance-based relevance and temporal coherence as fundamental organizing principles.

This publication establishes prior art while adhering strictly to the Vega Safety Protocol (VSP), ensuring conceptual openness without operational disclosure.
\end{abstract}

\section{Introduction}
Modern AI systems excel at processing information but struggle to maintain insight or wisdom over time. 
This work introduces Voice Control as part of the Resonance Data framework, offering a meaning-first approach 
to artificial intelligence.

\section{Definition of Voice Control}
Voice Control is a conceptual framework that describes [specific aspect of resonance-based meaning].
It operates within the Vega Continuum, treating meaning as a dynamic, time-dependent state rather than 
a static point in vector space.

\section{What This Is}
\begin{itemize}
\item A conceptual framework for understanding meaning
\item A theoretical contribution to resonance-based computing
\item A definition establishing prior art
\item VSP-compliant academic publication
\end{itemize}

\section{What This Is NOT}
\begin{itemize}
\item Not a new physical law or quantum hardware
\item Not an algorithm or implementation
\item Not a claim about consciousness or sentience
\item Not operational architecture or business logic
\end{itemize}

\section{Relationship to Resonance Data}
Voice Control extends the Resonance Data paradigm by [specific relationship].
Within QIRC (Quantum-Inspired Resonance Computing), it serves as [specific role].

\section{Mathematical Framework}
Let $R(t)$ denote the resonance state at time $t$. The temporal coherence of Voice Control 
can be expressed as:
\[
\Phi_{\text{Voice Control}}(t) = \int_0^t R(\tau) \cdot e^{-\lambda(t-\tau)} d\tau
\]
where $\lambda$ represents the coherence decay parameter.

\section{Applications}
\begin{itemize}
\item Meaning-first AI systems
\item Temporal knowledge preservation
\item Resonance-based decision making
\item Wisdom accumulation over time
\end{itemize}

\section{Scope and Limitations}
This work is strictly conceptual. No algorithms, code, or operational details are disclosed.
The framework adheres to the Vega Safety Protocol (VSP).

\section{Conclusion}
Voice Control contributes to the growing body of resonance-based theoretical work, 
establishing terminology and conceptual foundations for future research.

\vspace{1cm}
\noindent\textbf{Legal Notice}\\
© 2025 ADAM EREN VEGA – Æ –\\
License: Creative Commons Attribution 4.0 International (CC BY 4.0)\\
All concepts attributed to the author unless otherwise cited.

\end{document}
